\documentclass{scrreprt}
\usepackage{listings}
\usepackage{underscore}
\usepackage[bookmarks=true]{hyperref}
\usepackage[T1]{fontenc}
\usepackage[polish]{babel}
\usepackage[utf8]{inputenc}
\usepackage{lmodern}
\usepackage{graphicx}
\usepackage{tabularx}
\usepackage{booktabs}
\usepackage{enumitem}

\usepackage[table]{xcolor}
\usepackage[margin=1in]{geometry}
\usepackage{enumitem}
\usepackage{setspace}

\setlist{nolistsep}
\definecolor{green}{HTML}{66FF66}
\definecolor{myGreen}{HTML}{009900}
\definecolor{myBlue}{HTML}{003399}

\setlength{\parindent}{0pt}
\setlength{\parskip}{1ex plus 0.5ex minus 0.2ex}


\selectlanguage{polish}
\hypersetup{
    bookmarks=false,    % show bookmarks bar?
    pdftitle={Specyfikacja wymagań},    % title
    pdfauthor={Marta Wisniewska},                     % author
    pdfsubject={TeX and LaTeX},                        % subject of the document
    pdfkeywords={TeX, LaTeX, graphics, images}, % list of keywords
    colorlinks=true,       % false: boxed links; true: colored links
    linkcolor=blue,       % color of internal links
    citecolor=black,       % color of links to bibliography
    filecolor=black,        % color of file links
    urlcolor=purple,        % color of external links
    linktoc=page            % only page is linked
}%
\def\myversion{1.0 }
\date{}
%\title
\usepackage{hyperref}
\begin{document}

\begin{flushright}
    \rule{16cm}{5pt}\vskip1cm
    \begin{bfseries}
        \Huge{SPECYFIKACJA WYMAGAŃ}\\
        \vspace{1.9cm}
       Systemu zarządzania obrazami \\ do wspomagania pracy lekarzy \\  Image Manager 1.0\\
        \vspace{8.9cm}
 \end{bfseries}
        \Huge{{Marta Wi\'sniewska}}\\
        \vspace{1.9cm}
        \today\\
   
\end{flushright}

\tableofcontents

\chapter{Wstęp}
\section{Cel}
Niniejsza specyfikacja opisuje wymagania funkcjonalne i niefunkcjonalne wersji~1.0~systemu~zarządzania~obrazami do~wspomagania pracy lekarzy "Image Manager". Dokument służy do weryfikacji poprawno\'sci działania systemu. Jest przeznaczony dla użytkowników systemu oraz programistów, którzy będą rozwijać i implementować system w~przyszło\'sci.
\section{Zakres projektu}
System "Image Manager"~pozwala lekarzom na bezpieczne zarządzanie obrazami medycznymi. Umożliwia wczytywanie zdjęć, a następnie wy\'swietlanie ich w przeglądarce. System daje uprawnionym użytkownikom możliwo\'sć edycji obrazów poprzez dodawanie i~aplikowanie efektów do zdjęć. Zmodyfikowany obraz będący wynikiem działania programu, może zostać zapisany, a~następnie wykorzystany przez lekarza przy analizie i diagnozie lekarskiej.
\chapter{Opis ogólny}
\section{Perspektywa produktu}
System zarządzania obrazami do wspomagania pracy lekarzy jest nowym systemem. Diagram kontekstowy z rysunku \ref{fig:diag1} przedstawia proces komunikacji obiektów zewnętrznych z systemem wersji 1.0. System może być rozbudowany w przyszło\'sci o dodatkowe funkcjonalno\'sci takie jak możliwo\'sć dodawania znaczników do zdjęć i eksport danych o znacznikach obrazu do plików.
\begin{figure}[h]
    \centering
    \includegraphics[width=0.8\textwidth]{diagramKontekstowy.png}
    \caption{Diagram kontekstowy systemu Image Manager 1.0}
    \label{fig:diag1}
\end{figure}
\section{ Klasy oraz charakterystyki użytkowników }
\begin{center}
\begin{tabularx}{\textwidth}[h]{XX}
\arrayrulecolor{myBlue}\hline
\textbf{\textcolor{myBlue}{Klasa Użytkownika}} & 
\begin{minipage}[t]{\linewidth}%
\textbf{\textcolor{myBlue}{Opis}}
\end {minipage}\\

\hline
Użytkownik niezalogowany & 
\begin{minipage}[t]{\linewidth}%
Użytkownik nieposiadający uprawnień do~systemu. Jedyną udostępnioną funkcją jest logowanie.   
\end{minipage}\\
\arrayrulecolor{black}\hline
Lekarz & 
\begin{minipage}[t]{\linewidth}%
Zalogowany użytkownik systemu posiadający dostęp do obrazów. Uprawniony do korzystania z systemu w zakresie dodawania nowych obrazów, wy\'swietlania obrazów w przeglądarce oraz usuwania wybranych zdjęć. 
\end{minipage}\\
\arrayrulecolor{black}\hline
Specjalista & 
\begin{minipage}[t]{\linewidth}%
Zalogowany użytkownik systemu, posiadający wszystkie uprawnienia lekarza oraz dodatkowo możliwo\'sć edycji zdjęć, dodawania nowych efektów oraz zapisu zmodyfikowanych obrazów.  
\end{minipage}\\
\arrayrulecolor{black}\hline



\end{tabularx}
\end{center}

\newpage
\section{ Środowisko robocze}
Wymagania programowe stacji roboczej:\\
System operacyjny - dowolny system operacyjny obsługujący minimum jedną z niżej opisanych przeglądarek internetowych.\\

\textbf{\textcolor{myBlue}{SR-1}} . System powinien poprawnie działać w następujących przeglądarkach internetowych:
\begin{itemize}[label={--}]
\item{Internet Explorer 10 i nowsze}
\item{Mozilla Firefox 56 (najnowsza wersja)}
\item{Google Chrome 62 (najnowsza wersja)}
\item{Safari 7 i nowsze}
\end{itemize}
Zalecane są najnowsze wersje wymienionych wyżej przeglądarek.\\

Do poprawnego działania aplikacja wymaga, aby przeglądarka internetowa miała uaktywnione takie opcje jak: 
\begin{itemize}[label={--}]
\item{Obsługę języka „JavaScript”}
%\item{Akceptowanie ciasteczek ze strony z adresem zainstalowanej aplikacji.}
%\item{Obsługę protokołu „SSL 3.0”, bądź „TLS 1.0, 1.1, 1.2”.}
\end{itemize} 

%\textbf{\textcolor{myBlue}{SR-2}}. System  będzie działać na serwerze Amazon...
\section{Ograniczenia projektu i implementacji} 
\textbf{\textcolor{myBlue}{OP-1}}. Kod HTML powinien być zgodny ze standardem HTML 5.0. \\
\textbf{\textcolor{myBlue}{OP-2}}. Kod CSS powinien być zgodny ze standardem CSS 3.0. \\
\textbf{\textcolor{myBlue}{OP-3}}. Kod JS powinien być zgodny ze standardem ECMAScript 6. \\
\textbf{\textcolor{myBlue}{OP-4}}. Kod TS powinien być zgodny ze standardem TypeScript 2.6 \\
\chapter{Funkcjonalno\'sć systemu}
Wymagania funkcjonalne systemu zostały opracowane na podstawie diagramu przypadków użycia przedstawionego na rysunku \ref{fig:uc1}. Grupa przypadków użycia wyróżniona granatowym obramowaniem dotyczy  przypadków użycia lekarza (oraz specjalisty). Są to podstawowe przypadki użycia wersji 1.0 systemu "Image Manager".
\begin{figure}[h]
    \centering
    \includegraphics[width=0.99\textwidth]{useCases.png}
    \caption{Diagram przypadków użycia systemu Image Manager 1.0}
    \label{fig:uc1}
\end{figure}
	\section{Wymagania funkcjonalne}
	\begin{center}
		\begin{tabularx}{\textwidth}[t]{XX}
		\arrayrulecolor{black}
		\hline
		\textbf{\linespread{1.5}\textcolor{black}{Logowanie do systemu}} & \\
		\hline
		\textbf{\textcolor{myBlue}{WF-1}}
		\hspace{1cm}
			\begin{minipage}[t]{1.7\linewidth}%
			 System powinien umożliwiać logowanie użytkownika do systemu.
			\end{minipage}\\
		\textbf{\textcolor{myBlue}{WF-2}}
		\hspace{1cm}
			\begin{minipage}[t]{1.7\linewidth}%
			 System powinien przypisać użytkownikowi rolę i udostępnić tej roli okre\'slone uprawnienia w aplikacji.
			\end{minipage}\\
		\textbf{\textcolor{myBlue}{WF-3}} 
		\hspace{1cm}
			\begin{minipage}[t]{1.7\linewidth}%
			 W przypadku błędu podczas logowania system powinien wy\'swietlać komunikat o niepowodzeniu i przyczynie błędu.
			\end{minipage}\\
		
		\hline
		\textbf{\textcolor{black}{Dodawanie nowego obrazu}} \\
		\hline
		\textbf{\textcolor{myBlue}{WF-4}}
		\hspace{1cm}
			\begin{minipage}[t]{1.7\linewidth}%
			Zalogowani użytkownicy z rolą lekarza lub specjalisty powinni mieć możliwo\'sć dodawania obrazów z tytułem i opisem poprzez pola formularza. Użytkownik po kliknięciu przycisku "Upload"~powinien móc załadować dowolny obraz w~formacie: .png, .bmp, .gif, .jpg. Pola tekstowe powinny umożliwiać dodanie tytułu oraz opisu załadowanego obrazu.
			\end{minipage}\\
		\textbf{\textcolor{myBlue}{WF-5}}
		\hspace{1cm}
			\begin{minipage}[t]{1.7\linewidth}%
			System powinien wy\'swietlać obraz załadowany przez użytkownika.
\end{minipage}\\

\textbf{\textcolor{myBlue}{WF-6}}
\hspace{1cm}
\begin{minipage}[t]{1.7\linewidth}%
Użytkownik powinien mieć możliwo\'sć usunięcia załadowanego obrazu i~zmiany, przed dodaniem go z tytułem i~opisem do~systemu. 
\end{minipage}\\

\textbf{\textcolor{myBlue}{WF-7}}
\hspace{1cm}
\begin{minipage}[t]{1.7\linewidth}%
Po poprawnym dodaniu obrazu system powinien przechowywać obraz, jego tytuł, opis oraz datę dodania.
\end{minipage}\\


\textbf{\textcolor{myBlue}{WF-8}}
\hspace{1cm}
\begin{minipage}[t]{1.7\linewidth}%
W przypadku niepowodzenia procesu dodawania zdjęcia system powinien wy\'swietlać komunikat z przyczyną błędu.
\end{minipage}\\
\hline

\textbf{\textcolor{black}{Wy\'swietlanie dostępnych obrazów}} \\
\hline
\textbf{\textcolor{myBlue}{WF-9}}
\hspace{1cm}
\begin{minipage}[t]{1.7\linewidth}%
System powinien wy\'swietlać uprawnionemu użytkownikowi dostępne obrazy w postaci listy. Elementy listy powininny zawierać miniaturę obrazu, tytuł oraz datę dodania obrazu do systemu.
\end{minipage}\\

\textbf{\textcolor{myBlue}{WF-10}}
\hspace{0.8cm}
\begin{minipage}[t]{1.7\linewidth}%
System powinien umożliwić sortowanie listy dostępnych obrazów według: nazwy obrazu lub daty dodania obrazu.
\end{minipage}\\

\textbf{\textcolor{myBlue}{WF-11}}
\hspace{0.8cm}
\begin{minipage}[t]{1.7\linewidth}%
Po wybraniu elementu z listy system powinien wy\'swietlić obraz w pełnym rozmiarze wraz z informacjami dotyczącymi danego obrazu: tytułem, opisem, datą dodania obrazu.
\end{minipage}\\
\hline
\textbf{\textcolor{black}{Usuwanie wybranego obrazu}} \\
\hline
\textbf{\textcolor{myBlue}{WF-12}}
\hspace{0.8cm}
\begin{minipage}[t]{1.7\linewidth}%
Użytkownik powinien mieć możliwo\'sć usunięcia wybranego obrazu z listy dostępnych obrazów poprzez kliknięcie ikony 'X' przy wybranym obrazie. 
\end{minipage}\\

\textbf{\textcolor{myBlue}{WF-13}}
\hspace{0.8cm}
\begin{minipage}[t]{1.7\linewidth}%
System powinien wymagać potwierdzenia przez użytkownika akcji usunięcia zdjęcia.
\end{minipage}\\

\hline
\textbf{\textcolor{black}{Edycja obrazów}} \\
\hline

\textbf{\textcolor{myBlue}{WF-14}}
\hspace{0.8cm}
\begin{minipage}[t]{1.7\linewidth}%
Specjalista powinien móc zmodyfikować zdjęcie, wybierając jedną z opcji: Crop, Brightness Correction, Contrast Adjustment, Color Adjustment, Zoom.
\end{minipage}\\

\textbf{\textcolor{myBlue}{WF-15}}
\hspace{0.8cm}
\begin{minipage}[t]{1.7\linewidth}%
Specjalista powinien mieć możliwo\'sć dodawania nowych efektów do modyfikacji obrazów.
\end{minipage}\\

\textbf{\textcolor{myBlue}{WF-16}}
\hspace{0.8cm}
\begin{minipage}[t]{1.7\linewidth}%
Specjalista powinien móc odrzucić wprowadzone efekty edycji obrazu.
\end{minipage}\\


\textbf{\textcolor{myBlue}{WF-17}}
\hspace{0.8cm}
\begin{minipage}[t]{1.7\linewidth}%
Specjalista powinien mieć możliwo\'sć zapisu zmodyfikowanego obrazu.
\end{minipage}\\
\hline
\end{tabularx}
\end{center}

%\chapter{Wymagania dotyczące danych}

%\section{Pozyskiwanie, integralność, przechowywanie i usuwanie danych}

\chapter{Atrybuty jako\'sciowe}

\section{Bezpieczeństwo}
\textbf{\textcolor{myBlue}{BEZ-1}} Tylko zalogowani użytkownicy mogą zarządzać obrazami w systemie. \\ 
\textbf{\textcolor{myBlue}{BEZ-2}} Wszystkie dane dotyczące obrazów powinny być szyfrowane za pomocą protokołu TLS~1.2. \\ 
\textbf{\textcolor{myBlue}{BEZ-3}} Operacje modyfikacji obrazów mogą być wykonywane tylko przez autoryzowanych specjalistów. \\ 




\end{document}